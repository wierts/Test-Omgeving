\section{\glsentrylong{IBV}}

Context and institutions are the prevalent terms when it comes to~\ibv. %\rbv and \inbv.
This \ibv considers not only the firms (similar to~\cite{Porter:1980}) and resources (similar to~\cite{Barney:1991}) but also engages in considering the institutional constraints.\\
The lack thereof of considering institutions and context is part of the critiques in that have emerged~\cite{Narayanan:2005}.
Under certain circumstances for example, the pursuit of cost leadership can be deemed unethical in that the raising broilers however sustainable were seen as cruel by modern animal welfare standards. 
Some times the cost-leadership strategy can drive companies to engage in (illegal) price fixing actions. 
In the Dutch mobile phone market, the telecom providers (KPN, Vodafone and T-mobile) have been reprimanded twice in the last decade or so by the \gls{ACM}\fn{The \gls{ACM} is the new name for the \gls{NMA} witch is the Dutch regulatory agency for competitions, comparable tot the British \gls{OFT} (soon to be \gls{CMA}) } for price fixes deals on mobile calling costs\fn{Sourced from \url{http://www.volkskrant.nl/vk/nl/2844/Archief/archief/article/detail/3067315/2011/12/07/Mobiel-bellen-blijkt-te-duur-door-kartel.dhtml}}. \\
\cite{Kraaijenbrink:2009} also summarised \gls{RBV} critiques in his 2009 paper. 
He concluded that it is mainly the definition of `resource' and `valuable' in combination with the lack of acknowledgement of the combination of bundling resources and the human involvement, that is undermining strength of \gls{RBV}.
Likewise~\cite{Priem:2001} concluded that the contexts are missing from \gls{RBV}, where~\cite{Dung:2012} states that the ``resource-based view often neglects the issues of strategy implementation, i.e., various activities through which competitive advantage is directly created. Some resources may be valuable and rare at some point in time, this can change in an instant." 
Take a look at the operating system that Nokia used in its mobile phones in the early 2000s. 
This was considered the peak of user-friendliness; until the iPhone came along. 
The rare resource of a user-friendly operating system and user interface became defacto obsolete hence non-valuable.\\

The internal forces that underpin~\gls{IBV} are a reaction to criticisms on the theory of~\gls{InBV} (and~\rbv) of a lack of awareness of context~\cite{Narayanan:2005}.  
 %%% INSTITUTIONAL BASED VIEW EXPLAIN
The \ibv dictates that firms performance and choices do not only depend on resources and the industry the firm is competing in, but also depends on the (a) environment (institutional constraints) in which managers and firms pursue their interest~\cite{Peng:2008b}.%
The institutional framework can have a positive effect on innovations (in the US for example) and a negative effect in Japan. 
In this case old drug are more profitable than new drugs in Japan~\cite{Peng:2008b}. \\
On the other hand~\ibv proposed that (b) formal and informal institutions combine to govern firm behaviour, in situations where formal constraints fail, informal constraints play a larger role in reducing uncertainty and providing consistency to managers and firms~\cite{Peng:2008b}. \\
Both effects (a) and (b) can be summarised in the word `Context'. Context is the third leg~\cite{Peng:2009} that influences the various decisions that firms have decide on in~\ib. 
The institutions (which are part of the context) present themselves in two forms `formal' and `informal'.~\cite{Peng:2002}. The latter are things like accepted social behaviour and come into play when formal constraints fail~\cite{North:1990,DiMaggio:1983,Scott:1995}. The first include political rules, judicial decisions, and economic contracts. More on the theory of institutions will be investigated in section~\ref{sec:InTh}.\\
So~\ibv~takes into account not only strategic choices driven by industry conditions and firm-specific resources, that traditional strategy research emphasises (~\cite{Porter:1980,Barney:1991}), but are also a reflection of the formal and informal constraints of a particular institutional framework that decision makers confront (\cite{Oliver:1997,Scott:1995}). Given the influence of institutional frameworks on firm behaviour, any strategic choice that firms make is inherently affected by the formal and informal constraints of a given institutional framework (\cite{North:1990,Oliver:1997}). \\
\begin{figure}[h]

\centering
\begin{tikzpicture}
% STYLES
\tikzset{
    ellips/.style={ellipse, minimum width=100pt,
    align=center,node distance=3cm,inner sep=5pt,text width=2.5cm,minimum    
    height=2.0cm,>=stealth'}%fill=black!10
}

\node [ellips](Choices) {Strategic Choices};
\node [above=2cm of Choices](dummy) {};
\node [ellips, force, left=1.5cm of dummy] (Institutions) {Institutions};
\node [ellips, force, right=1.5cm of dummy,drop shadow,draw] (Organisations) {Organisations};

% Draw the links between 
\path[<->,thick] 
   (Choices) edge node[anchor=center, text width=3.5cm, below left, midway] {Industry conditions and 
    firm-specific resources}  (Institutions) 
   (Choices) edge node[anchor=center, text width=3.5cm, below right, midway] {Formal and informal 
   constraints} (Organisations)
 (Institutions) edge  node [midway, below] {interaction} node[midway, above] {Dynamic} (Organisations);

\end{tikzpicture} 
\label{fig:Peng2000} 
\caption{Institutions, organisations, and strategic choices. Source:~\cite{Peng:2000}}%
\end{figure}

So \ibv focusses not only on strategy and the firms that make these strategic choices but takes into account the institutions that govern the playing field. Moreover the interaction between the firms, institutions and the strategic choices is what \ibv is all about.\\
Strategic literature does not discuss the specific relationship between strategic choices and institutional frameworks~\cite{Peng:2008}.
In contrast to earlier theories,~\ibv does not exist on it's own. It is merely an extension on earlier theories.This figure (\ref{fig:Peng2000}) shows the dependance on both the theories of~\cite{Barney:2001} and~\cite{Porter:1980} for~\ibv. 
Obviously~\ibv is not an attempt to dismiss other theories more an attempt to complete theories on strategy as they exist at the moment. 
Strategy is about making the right choices at the correct moment. \\
\begin{figure}[h]
\centering
\begin{tikzpicture}[scale=0.7, transform shape]
% STYLES
\tikzset{
    ellips1/.style={ellipse, draw, 
    align=center, node distance=2.5cm, inner sep=5pt, >=stealth,text width=4.0cm},
  %fill=black!10,minimum ,text width=3.3cm, height=2.75cm, text width=2.9cm,
    ellips/.style={ellipse, draw, minimum width=100pt,
    align=center,node distance=3cm,inner sep=5pt, text width=2cm,   
   minimum height=1.0cm,>=stealth}%fill=black!10, 
}
\node [ellips1](Industry) {Industry-Based Competition};
\node [ellips1, below=1cm of Industry](Firm) {Firm Specific Resources and Capabilities};
\node [ellips1, below=1cm of Firm] (Institutional) {Institutional Conditions and Transitions};
\node [ellips, right=1.5cm of Firm] (Strategy) {Strategy};
\node [ellips, right=1.5cm of Strategy] (Performance) {Performance};

% Draw the links between 
\path[->,thick] 
   (Industry.east) edge  (Strategy) 
   (Firm.east) edge  (Strategy)
   (Institutional.east) edge  (Strategy)
   (Strategy) edge  (Performance);
 
\end{tikzpicture} 
\caption{The Institution-Based View as a Third Leg for a Strategy Tripod. Source:~\cite{Peng:2009}}%
\label{fig:Peng2009} 
\end{figure}

Treating institutions as independent variables, an institution-based view on business strategy, therefore, focuses on the dynamic interaction between institutions and organisations, and considers strategic choices as the outcome of such an interaction (see figure~\ref{fig:Peng2000})~\cite{Peng:2002}.
Not only have more scholars come to realise that institutions matter~\cite{Powell:1991,Scott:1995}, but also that strategy research cannot just focus on industry conditions and firm resources.~\cite{Khanna:1997}\\


