\chapter{Literature Review}

\newacronym{RBV}{RBV}{Resourced Based View}
\newacronym{IBV}{IBV}{Institutional Based View}
\newacronym{IB}{IB}{International business}
\newacronym{IP}{IP}{Intellectual Property}
\newacronym{MNE}{MNE}{Multinational Enterprises}

In an time of austerity and double dips in the economy, \gls{IB} is playing the game on different fields. Companies like Apple and Samsung are not only fighting for customers but also fighting in court over\gls{IP} as well. Next to this fight over \gls{IP} subsidies are at the forefront of the public debate as well. Boeing and Airbus have been fighting over subsidies for decades where on more than one occasion the \gls{WTO} has ruled on the validity of these subsidies. More recently solar panels have become a topic of tariffs between the European Union and China. \\
The field of play is governed by governments, trade blocks and the \gls{WTO} on the one hand and \gls{IB} on the other. a Multitude of forces are acting on this playing field and the different actors on this pitch have to cooperate. Different \gls{MNE} will cope differently with the challenges that are set by the institutions and the environment that they are operating in. That this environment is of importance is explained by \gls{IBV}~\cite{Kostova:1999,Meyer:2009,Wang:2012} 

\section{\glsdesc{IBV} in international Business}

\gls{IBV} has been a reaction on the theory of \gls{RBV} introduced by  \cite{Barney:1991}. The \gls{RBV} theory has gained a lot of support in the international business community. The aforementioned article has received more than 5000 citations\footnote{according to Web of Knowledge %($$http://cel.webofknowledge.com/InboundService.do?%product=CEL&SID=N1%40KDj7JAHJp54a7poE&UT=A1991FE14500007&SrcApp=PARTNER_APP&Init=Y%es&action=search&Func=Frame&SrcAuth=stanwire&customersID=stanwire&viewType=summary&IsPro%ductCode=Yes&mode=CitingArticles$$)
} since it has been published in 1991.\\ Where \gls{RBV} stated that a firms strategic advantage is depended on it's heterogeneous resources (a bundle of all assets, knowledge, and capabilities) which have to be ``
(a) must be valuable, in the sense that it exploit opportunities and/or neutralises threats in a firm’s environment, (b) must be rare among a firm’s current and potential competition, (c) must be imperfectly imitable, and (d) there cannot be strategically equivalent substitutes for this resource that are valuable but neither rare or imperfectly imitable'' \cite{Barney:1991} otherwise known as VRIN.

Where \gls{RBV}  is introspective in nature \gls{IBV} is extrospective.


The central argument is that “organisations conform to the rules and beliefs systems in the environment because this isomorphism (regulatory, cognitive and normative) earns them legitimacy


Not only have more scholars have come to realise that institutions matter and, that strategy research cannot just focus on industry conditions and firm resources alone \cite{Powell:1991,Scott:2001}.

Nowadays institutional theory appears to be a highly insightful approach when probing into organisational strategies in Asia~\cite{Hoskisson:2000}


\Gls{IB} has long been a favourite topic in academic literature. The term \gls{IB} alone gives over 2M hits in Google Scholar. Since~\cite{Porter:1980} introduced the concept of favourable industries, 

When introduced the \gls{RBV} in international business literature this view gained a lot of support. 
The theory has been expanded upon and \gls{IBV} was introduced by \cite{Kostova:1999,Meyer:2009,Wang:2012}