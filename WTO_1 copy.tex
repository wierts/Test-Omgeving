\newacronym{gats}{GATS}{General Agreement on Trade in Services}
\chapter{WTO}

\section{WTO purpose}
The way to view the \wto is not as obvious as one might think on first hand.

\subsection{Negotiating forum}


\subsection{Trade Agreements}
Before the time of the \wto agreements have been reached during the \gls{gats} era. The results of these was a number of agreements regarding the trading of Goods during 1947--1994 trade talk rounds. Not only were agreements reached on trading of goods but also on lower customs duty rates and other trade barriers.

Since the inception of the \wto the new rules have been committed in the \gls{gatt}. The \wto has also been active in settings rules for a number of intellectual property such as copyrights, patents, trademarks, geographical names used to identify products.

\subsection{Set of rules}




\subsection{Dispute Settlement}

When, albeit the negotiated agreements, necessary members can bring disputes before the \wto.
Settling these disputes is the pillar of the \wto trading system. The rules set by the \wto are not as effective when there is no system to enforce these rules. The set of rules is not designed to pass judgement, the priority is to settle disputes (through consultations if possible). 
 
In 2008 only about 136 of the nearly 369 cases had reached the full panel process. Most of the rest have either been notified as settled ``out of court'' or remain in a prolonged consultation phase -- some since 1995 \cite{WTO_History}.

\subsection{}