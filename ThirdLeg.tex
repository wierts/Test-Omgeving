\begin{figure}[h]
\centering
\begin{tikzpicture}[scale=0.7, transform shape]
% STYLES
\tikzset{
    ellips1/.style={ellipse, draw, 
    align=center, node distance=2.5cm, inner sep=5pt, >=stealth,text width=4.0cm},
  %fill=black!10,minimum ,text width=3.3cm, height=2.75cm, text width=2.9cm,
    ellips/.style={ellipse, draw, minimum width=100pt,
    align=center,node distance=3cm,inner sep=5pt, text width=2cm,   
   minimum height=1.0cm,>=stealth}%fill=black!10, 
}
\node [ellips1](Industry) {Industry-Based Competition};
\node [ellips1, below=1cm of Industry](Firm) {Firm Specific Resources and Capabilities};
\node [ellips1, below=1cm of Firm] (Institutional) {Institutional Conditions and Transitions};
\node [ellips, right=1.5cm of Firm] (Strategy) {Strategy};
\node [ellips, right=1.5cm of Strategy] (Performance) {Performance};

% Draw the links between 
\path[->,thick] 
   (Industry.east) edge  (Strategy) 
   (Firm.east) edge  (Strategy)
   (Institutional.east) edge  (Strategy)
   (Strategy) edge  (Performance);
 
\end{tikzpicture} 
\caption{The Institution-Based View as a Third Leg for a Strategy Tripod. Source:~\cite{Peng:2009}}%
\label{fig:Peng2009} 
\end{figure}
