\usepackage[utf8]{inputenc} %f ̧r Windows %direct gebruik van accenten
\input{ix-utf8enc.dfu}% dit is om u ̃ te laden
\usepackage[T1]{fontenc}
\usepackage[small,normal,bf,up]{caption}
\usepackage{eurosym,supertabular,multirow,tabularx} %,bigstrut,
\usepackage{geometry}
\usepackage{url} % om urls goed weer te geven in voetnoten
\usepackage[toc,page,title]{appendix}
\usepackage[multiple,flushmargin]{footmisc}
%\usepackage{cprotect} %gebruikt om verbatim in footnotes te zetten
\usepackage{hanging}% provides the \hangpara command
\usepackage{enumerate}
\usepackage[acronym,toc]{glossaries}  
\usepackage{tikz} % tekenen van plaatjes
\usetikzlibrary{arrows,shapes,decorations.pathmorphing,backgrounds,fit,positioning,shapes.symbols,chains}
\glsdisablehyper%
\makeglossaries%
\doublespacing%

\DeclareUnicodeCharacter{0169}{\~u} % make the character known


%--------------Layout van A4 document-----------------
\geometry{
  includeheadfoot,
  margin=2.5cm,
  hdivide={ ,19cm, }  
}

%------------------Footnotes--------------------------
\renewcommand{\footnotemargin}{1em}
% changes the above and sets the footnote mark just right of the left margin border.

\newcommand{\fn}[1]{\footnote{\hangpara{3em}{1} #1}}
% makes a new footnote command \fn{} with a hanging indent of 3em (hanging indent starts after the first line)

%------------Biblatex Optie 1---------------------------

%%\usepackage[backend=biber,style=authoryear-luh-ipw,isbn=false,url=false]%{biblatex} ,style=historian ,citestyle=apa
\usepackage[
backend=biber,
citestyle=authoryear-comp,
bibstyle=authortitle,
sorting=nyt,
bibencoding=utf8,
dashed=false,
maxcitenames=1,
isbn=false,
url=false,
babel=hyphen,
hyperref=true,
doi=false]
{biblatex}


%--------Biblatex-------------------------------------

\DeclareNameAlias{sortname}{first-last}

\DeclareCiteCommand{\cite}[\mkbibbrackets]
  {\usebibmacro{prenote}}
  {\usebibmacro{citeindex}%
   \usebibmacro{cite}}
  {\multicitedelim}
  {\usebibmacro{postnote}}

\DeclareCiteCommand*{\cite}[\mkbibbrackets]
  {\usebibmacro{prenote}}
  {\usebibmacro{citeindex}%
   \usebibmacro{citeyear}}
  {\multicitedelim}
  {\usebibmacro{postnote}}


\newcounter{mymaxcitenames}
\AtBeginDocument{%
  \setcounter{mymaxcitenames}{\value{maxnames}}%
}

\renewbibmacro*{begentry}{%
  \printtext[brackets]{%
    \begingroup%
    \defcounter{maxnames}{\value{mymaxcitenames}}%
    \printnames{labelname}%
    \setunit{\nameyeardelim}%
    \usebibmacro{cite:labelyear+extrayear}%
    \endgroup%
    }%
  \quad% or \addspace
}

%--------END Biblatex-------------------------------------



%------------Biblatex Optie 2---------------------------

\input{BiblatexOptie2}
%----------------------------CSQuotes na Biblatex
\usepackage[babel]{csquotes}%
\usepackage[english]{babel}%
%------------------ FONT ----------------------------------------


% ALTERNATIVE
%\usepackage{lmodern}
%\usepackage[urw-garamond]{mathdesign}
%\usepackage[math]{iwona}
%\usepackage[default]{lato}
%\usepackage{venturis2}
\usepackage[sf]{quattrocento}
%\usepackage{libertine}
%\usepackage[T1]{fontenc}

%------------------ END FONT ------------------

\linespread{1.5} %\setlength{\parskip}{\baselineskip}
\setlength{\headheight}{43,3pt}
\setlength{\voffset}{0pt}

\pagestyle{fancy}
\fancyhf{}     % clear header & footer
\fancyhead[C]{\small\sf\nouppercase{\leftmark}}
\fancyhead[L]{\includegraphics[width=.19\textwidth]{ThesisFigs/uva_logo}} %links tu logo
%\fancyhead[R]{thepage} %rechts Abbott logo \includegraphics[width=.2\textwidth]{ThesisFigs/abbott}
%\fancyfoot[L]{\thepage}
\fancyfoot[R]{\thepage}

\fancypagestyle{plain}{\fancyhf{}\fancyfoot[R]{\thepage}
\renewcommand{\headrulewidth}{0.4pt}
\renewcommand{\footrulewidth}{0.4pt}}

\renewcommand{\captionfont}{\small\textit}

\renewcommand{\headrule}{{\color{grey}%
\hrule width\headwidth%
 height\headrulewidth \vskip-\headrulewidth}
 }
%\setlength{\footrulewidth}{\headrulewidth}
\renewcommand{\footrule}{{\color{grey}% 
  \vskip-\footruleskip\vskip-\footrulewidth%
\hrule width\headwidth height\footrulewidth\vskip\footruleskip}
}
\renewcommand{\chaptermark}[1]{\markboth{\textbf{\thechapter\ #1}}{}}
\renewcommand{\sectionmark}[1]{\markright{\thesection\ #1}{}}
\fancyheadoffset{0cm} % zorgt er voor dat de head en foot lijnen even breed zijn als de textbreedte


%--------title Opmaak------------------------

    \pdfinfo { /Title  (WTO LEadership)
               /Creator (TeX)
               /Producer (pdfTeX)
               /Author (Duco Wiertsema duco@wiertsema.net)
               /CreationDate (D:\today)  %format D:YYYYMMDDhhmmss
               /ModDate (D:20030815213532)
               /Subject (Writing a PhD thesis in LaTeX)
               /Keywords (PhD, Thesis)}
    \pdfcatalog { /PageMode (/UseOutlines)
                  /OpenAction (fitbh)  }


\title{Coevolution of the WTO and the and international business}
\subtitle{How do multiple embeddedness and coevolution effect the decision making of MNE wrt the WTO}

  \author{\href{mailto:duco@wiertsema.net}{Duco Wiertsema}}
  \university{\href{http://www.uva.nl}{University of Amsterdam}}
% insert below the file name that contains the crest in-place of 'UnivShield'
  \crest{\includegraphics[width=100mm]{uva_logo}}
%
\superA{}
\superB{}


% insert below the file name that contains the crest in-place of 'UnivShield'
% \crest{\IncludeGraphicsW{UnivShield}{40mm}{14 14 73 81}}
%
%\renewcommand{\submittedtext}{change the default text here if needed}
\degree{Master of Science in Business Studies}
\degreedate{June 2013}

% turn of those nasty overfull and underfull hboxes
\hbadness=10000
\hfuzz=40pt




%----------einde titel opmaak-------------------
%\floatplacement{table}{htbp} \floatplacement{figure}{htbp}
%\floatstyle{plaintop} \restylefloat{table}
%\renewcommand{\vec}{\boldsymbol}
%\newcommand{\unit}[1]{\ \mathrm{#1}}
%\newcommand{\srcinput}[2]{\noindent\framebox[\columnwidth]{\sc #2}{\scriptsize \verbatiminput{#1}}}
\setlength{\columnseprule}{\headrulewidth}
\setcounter{tocdepth}{2}

