\begin{figure}
\centering
\begin{tikzpicture}[scale=0.8, transform shape]
% STYLES
\tikzset{%
    force/.style={%
node distance = 1cm, 
%auto,
rectangle,
rounded corners, 
fill=black!10,
node distance=3cm,
inner sep=5pt, 
text width=4cm, 
text badly centered, 
minimum height=1.3cm
                          }
             }
% Draw forces
\node [force] (rivalry) {Rivalry among existing competitors};
\node [force, above of=rivalry] (substitutes) {Threat of substitutes};
\node [force, left=1cm of rivalry] (suppliers) {Bargaining power of suppliers};
\node [force, right=1cm of rivalry] (users) {Bargaining power of users};
\node [force, below of=rivalry] (entrants) {Threat of new entrants};

% Draw the links between forces
\path[->,thick] 
(substitutes) edge (rivalry)
(suppliers) edge (rivalry)
(users) edge (rivalry)
(entrants) edge (rivalry);
\end{tikzpicture} 
\caption{Five Forces Diagram source:~\cite{Porter:1980}}%
\label{fig:5forces} %
\end{figure}
