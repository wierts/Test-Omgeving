\begin{figure}[h]
\label{fig:5forces} 
\centering
\begin{tikzpicture}
[node distance = 1cm, auto,font=\footnotesize,
% STYLES
every node/.style={node distance=3cm},
% The comment style is used to describe the characteristics of each force
comment/.style={rectangle, inner sep= 5pt, text width=4cm, node distance=0.25cm, },%font=\scriptsize\sffamily
% The force style is used to draw the forces' name
force/.style={rectangle,rounded corners, draw, fill=black!10, inner sep=5pt, text width=4cm, text badly centered, minimum height=1.3cm}
] 

% Draw forces
\node [force] (rivalry) {Rivalry among existing competitors};
\node [force, above of=rivalry] (substitutes) {Threat of substitutes};
\node [force, left=1cm of rivalry] (suppliers) {Bargaining power of suppliers};
\node [force, right=1cm of rivalry] (users) {Bargaining power of users};
\node [force, below of=rivalry] (entrants) {Threat of new entrants};

%%%%%%%%%%%%%%%
% Change data from here


% Draw the links between forces
\path[->,thick] 
(substitutes) edge (rivalry)
(suppliers) edge (rivalry)
(users) edge (rivalry)
(entrants) edge (rivalry);

\end{tikzpicture} 
\caption{Five Forces Diagram source:~\cite{Porter:1980}}%
\end{figure}
