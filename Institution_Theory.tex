\section{Institutional Theory}\label{sec:InTh}

Following~\cite{North:1990}, we understand institutions as ``humanly devised constraints that structure human interaction``. 
Organisations can be seen as technical and social structures and processes. 
However they cannot be seen as rational~\cite{Westney:2005}. 
A basic premise of institutional theory is that the ``environment`` is socially organised. 
The point is organisations consist of relations not of transactions~\cite{Westney:2005}. This changes the dynamic of organisations.
Institutions are pervasive in that they are capable of shaping the behaviours of multiple organisations (i.e. individuals, firms, industries, and~\glspl{NGO})~\cite{Peng:2008}.  
\cite{North:1990} defines the same phenomena as:
Institutional factors function as the formal and informal ``rules of the game`` that socially constrain contracting practices between the \gls{BoD} and executives.

More broadly speaking, institutions serve to reduce uncertainty for different actors by conditioning the ruling norms of firm behaviours and defining the boundaries of what is considered legitimate~\cite{Peng:2008}.\\
The formal and informal institutions can be summarised as in table~\ref{tab:peng2008}
Firms do not only have to look at their resources and capabilities~\cite{Barney:1991}, but have to look at ``the rules of the game''~\cite{Scott:1995}. 
These so called rules include the environment that the firm \mne~has to adhere to.
Institutions are the formal and informal rules of the game~\cite{North:1990}. These institutions are influencing the decision making process in~\gls{IB}.

\begin{table}[htb]
  \centering
  \caption[Dimensions of Institutions]{Dimensions of Institutions. Source~\cite{Peng:2008b}} 
  \label{tab:peng2008}
\begin{tabular}{c|l|c} 
 % \toprule
  Degree of Formality & Examples & Supportive Pillars\\ 
  \midrule 
  \midrule
  Formal Institutions& Laws&  Regelatory (coercive)\\
  &Regulations&\\
  &Rules&\\
  Informal Institutions&Norms &Normative \\
  & Cultures&Cognitive\\
  &Ethics&\\ 
  \bottomrule
\end{tabular}
\end{table}

\noindent
The actions of institutions can be divided into three pillars. 
Compliance from the firms occurs through \\
(1) expedience (regulative or coercive pillar),\\
(2) social obligation (normative pillar), or \\
(3) on a taken for granted basis (cognitive pillar) \\
 
 The cognitive pillar mentioned as the mimetic \iso~where organisations respond to uncertainty by adopting patterns of other organisations that are deemed `successful'~\cite{Westney:2005,Peng:2008,Kostova:1999,DiMaggio:1983,Scott:1995}\\ 
\glspl{MNE}~conform to these pillars or isomorphisms because these provide legitimacy~\cite{Powell:1991}.
The terms pillar and \iso are used interchangeable in strategic literature. 
For the sake of clarity in here the term \iso will be used as defined as: ``corresponding or similar in form and relations''. Here the term is used to state that firms adopt similar structures and strategies.\\

The central argument with regard to institutions is that “organisations conform to the rules and beliefs systems in the environment because this isomorphism (regulatory, cognitive and normative) earns them legitimacy.
Not only have more scholars have come to realise that institutions matter and, that strategy research cannot just focus on industry conditions and firm resources alone~\cite{Powell:1991,Scott:1995}.
Nowadays institutional theory appears to be a highly insightful approach when probing into organisational strategies in Asia~\cite{Hoskisson:2000}.

%orgasational theory

\cite{Westney:2005} The Organisational field may be coterminous with industry, on the other hand it may be applied to a more circumscribed field such as regional economies or a range of organisations such as the Fortune 500.


 

%---------------- Aantekeningen---------------------------------------------------------

When markets work smoothly in developed economies, ``the market-supporting institutions are almost invisible``.~\cite{McMillan:2008}
The effect of institutions on strategy can be seen most obviously in the asian economies~\cite{Peng:2002}.
Where \rbv~looked solely at the firm in a set environment \ibv~also takes the surroundings into account. These surroundings are the institutions that govern the environment the \mne~is playing the game. \\

According to~\cite{Peng:2003} unfortunately, little is known about how organisations make strategic choices when confronting such large-scale institutional transitions.