\section{Institutional Theory}\label{sec:InTh}


Following~\cite{North:1990}, we understand institutions as ``humanly devised constraints that structure human interaction". 
Institutional factors function as the formal and informal ``rules of the game'' that socially constrain contracting practices between the \gls{BoD} and executives~\cite{North:1990}.
These formal en informal constraints operate in structures for both social and economic exchanges. 
Institutions are pervasive in that they are capable of shaping the behaviours of multiple actors (i.e. 
individuals, firms, industries, and~\glspl{NGO}). 
More broadly speaking, institutions serve to reduce uncertainty for different actors by conditioning the ruling norms of firm behaviours and defining the boundaries of what is considered legitimate~\cite{Peng:2008}.\\


According to~\cite{Peng:2003} unfortunately, little is known about how organisations make strategic choices when confronting such large-scale institutional transitions.


 Firms do not only have to look at their resources and capabilities~\cite{Barney:1991}, but have to look at ``the rules of the game''~\cite{Scott:1995}. These so called rules include the environment that the firm \mne~has to adhere to.\\

The central argument is that “organisations conform to the rules and beliefs systems in the environment because this isomorphism (regulatory, cognitive and normative) earns them legitimacy.
%losse informatie
Not only have more scholars have come to realise that institutions matter and, that strategy research cannot just focus on industry conditions and firm resources alone~\cite{Powell:1991,Scott:1995}.
%losse informatie
Nowadays institutional theory appears to be a highly insightful approach when probing into organisational strategies in Asia~\cite{Hoskisson:2000}.

When introduced the \gls{RBV} in international business literature this view gained a lot of support. 
The theory has been expanded upon and \gls{IBV} was introduced by~\cite{Kostova:1999,Meyer:2009,Wang:2012}.




